%% Generated by Sphinx.
\def\sphinxdocclass{report}
\documentclass[letterpaper,10pt,english]{sphinxmanual}
\ifdefined\pdfpxdimen
   \let\sphinxpxdimen\pdfpxdimen\else\newdimen\sphinxpxdimen
\fi \sphinxpxdimen=.75bp\relax

\PassOptionsToPackage{warn}{textcomp}
\usepackage[utf8]{inputenc}
\ifdefined\DeclareUnicodeCharacter
% support both utf8 and utf8x syntaxes
  \ifdefined\DeclareUnicodeCharacterAsOptional
    \def\sphinxDUC#1{\DeclareUnicodeCharacter{"#1}}
  \else
    \let\sphinxDUC\DeclareUnicodeCharacter
  \fi
  \sphinxDUC{00A0}{\nobreakspace}
  \sphinxDUC{2500}{\sphinxunichar{2500}}
  \sphinxDUC{2502}{\sphinxunichar{2502}}
  \sphinxDUC{2514}{\sphinxunichar{2514}}
  \sphinxDUC{251C}{\sphinxunichar{251C}}
  \sphinxDUC{2572}{\textbackslash}
\fi
\usepackage{cmap}
\usepackage[T1]{fontenc}
\usepackage{amsmath,amssymb,amstext}
\usepackage{babel}



\usepackage{times}
\expandafter\ifx\csname T@LGR\endcsname\relax
\else
% LGR was declared as font encoding
  \substitutefont{LGR}{\rmdefault}{cmr}
  \substitutefont{LGR}{\sfdefault}{cmss}
  \substitutefont{LGR}{\ttdefault}{cmtt}
\fi
\expandafter\ifx\csname T@X2\endcsname\relax
  \expandafter\ifx\csname T@T2A\endcsname\relax
  \else
  % T2A was declared as font encoding
    \substitutefont{T2A}{\rmdefault}{cmr}
    \substitutefont{T2A}{\sfdefault}{cmss}
    \substitutefont{T2A}{\ttdefault}{cmtt}
  \fi
\else
% X2 was declared as font encoding
  \substitutefont{X2}{\rmdefault}{cmr}
  \substitutefont{X2}{\sfdefault}{cmss}
  \substitutefont{X2}{\ttdefault}{cmtt}
\fi


\usepackage[Bjarne]{fncychap}
\usepackage{sphinx}

\fvset{fontsize=\small}
\usepackage{geometry}

% Include hyperref last.
\usepackage{hyperref}
% Fix anchor placement for figures with captions.
\usepackage{hypcap}% it must be loaded after hyperref.
% Set up styles of URL: it should be placed after hyperref.
\urlstyle{same}

\usepackage{sphinxmessages}
\setcounter{tocdepth}{1}



\title{recap.\@{} Spinx 1}
\date{Feb 29, 2020}
\release{}
\author{jk}
\newcommand{\sphinxlogo}{\vbox{}}
\renewcommand{\releasename}{}
\makeindex
\begin{document}

\pagestyle{empty}
\sphinxmaketitle
\pagestyle{plain}
\sphinxtableofcontents
\pagestyle{normal}
\phantomsection\label{\detokenize{index::doc}}


Contents:
\begin{quote}

test\_graphs mmvv
\end{quote}


\chapter{momo MY A}
\label{\detokenize{a:momo-my-a}}\label{\detokenize{a:lien-sur-a}}\label{\detokenize{a::doc}}
Kéroack, il neige.


\begin{savenotes}\sphinxatlongtablestart\begin{longtable}[c]{\X{1}{2}\X{1}{2}}
\hline

\endfirsthead

\multicolumn{2}{c}%
{\makebox[0pt]{\sphinxtablecontinued{\tablename\ \thetable{} -- continued from previous page}}}\\
\hline

\endhead

\hline
\multicolumn{2}{r}{\makebox[0pt][r]{\sphinxtablecontinued{Continued on next page}}}\\
\endfoot

\endlastfoot

\sphinxcode{\sphinxupquote{newclass}}
&

\\
\hline
\sphinxcode{\sphinxupquote{NewClass}}
&

\\
\hline
\end{longtable}\sphinxatlongtableend\end{savenotes}


\chapter{momo le B}
\label{\detokenize{b:momo-le-b}}\label{\detokenize{b::doc}}
Le tien est le mien, Quel grain de pollen ?


\section{des codes:}
\label{\detokenize{b:des-codes}}

\section{Table}
\label{\detokenize{b:table}}

\begin{savenotes}\sphinxattablestart
\centering
\begin{tabulary}{\linewidth}[t]{|T|T|T|}
\hline
\sphinxstartmulticolumn{2}%
\begin{varwidth}[t]{\sphinxcolwidth{2}{3}}
\sphinxstyletheadfamily Inputs
\par
\vskip-\baselineskip\vbox{\hbox{\strut}}\end{varwidth}%
\sphinxstopmulticolumn
&\sphinxstyletheadfamily 
Output
\\
\hline\sphinxstyletheadfamily 
A
&\sphinxstyletheadfamily 
B
&\sphinxstyletheadfamily 
A or B
\\
\hline\sphinxstartmulticolumn{2}%
\begin{varwidth}[t]{\sphinxcolwidth{2}{3}}
False
\par
\vskip-\baselineskip\vbox{\hbox{\strut}}\end{varwidth}%
\sphinxstopmulticolumn
&
False
\\
\hline
True
&
False
&
True
\\
\hline
False
&
True
&
True
\\
\hline\sphinxstartmulticolumn{2}%
\begin{varwidth}[t]{\sphinxcolwidth{2}{3}}
True
\par
\vskip-\baselineskip\vbox{\hbox{\strut}}\end{varwidth}%
\sphinxstopmulticolumn
&
True
\\
\hline
\end{tabulary}
\par
\sphinxattableend\end{savenotes}


\begin{savenotes}\sphinxatlongtablestart\begin{longtable}[c]{\X{1}{2}\X{1}{2}}
\hline

\endfirsthead

\multicolumn{2}{c}%
{\makebox[0pt]{\sphinxtablecontinued{\tablename\ \thetable{} -- continued from previous page}}}\\
\hline

\endhead

\hline
\multicolumn{2}{r}{\makebox[0pt][r]{\sphinxtablecontinued{Continued on next page}}}\\
\endfoot

\endlastfoot

\sphinxcode{\sphinxupquote{newclass}}
&

\\
\hline
\end{longtable}\sphinxatlongtableend\end{savenotes}

Une classe :py:class:JK\_py.newclass.NewClass is born.


\chapter{momo C is C}
\label{\detokenize{c:momo-c-is-c}}\label{\detokenize{c::doc}}
Un rat x mange des grains.

J’ai un lien sur “a” à écrire : {\hyperref[\detokenize{a:lien-sur-a}]{\sphinxcrossref{\DUrole{std,std-ref}{momo MY A}}}}.

Une autre avec “doc” : {\hyperref[\detokenize{a::doc}]{\sphinxcrossref{\DUrole{doc}{momo MY A}}}} or {\hyperref[\detokenize{a::doc}]{\sphinxcrossref{\DUrole{doc}{momo MY A}}}}, the link refers to a.

Une foute note : la note %
\begin{footnote}[1]\sphinxAtStartFootnote
Mon ami mon chum mon pote
%
\end{footnote} ?

une classe :py:class:kkoonn ss bv

\noindent\sphinxincludegraphics[width=200\sphinxpxdimen]{{jn_fonction}.png}

Le commentaire de cette image n’est pas une caption (avec.. figure).


\bigskip\hrule\bigskip


\begin{sphinxadmonition}{note}{Note:}
This is a paragraph
\end{sphinxadmonition}


\chapter{Graph test !}
\label{\detokenize{test_graphs:graph-test}}\label{\detokenize{test_graphs::doc}}
un graphe 01

\noindent\sphinxincludegraphics{{plantuml-e354b2b5b1642e56e81191c3c5cfca5b524873b1}.png}

\noindent\sphinxincludegraphics{{plantuml-53e5945050e133304bd9a2d21ed55ec810b12b59}.png}

\noindent\sphinxincludegraphics{{plantuml-77c01c64b1ad32366da25946dc15b7b881e97f2c}.png}

\noindent\sphinxincludegraphics{{plantuml-7bcddc0fdf59cfbd5e5b7fad7b3d012831c49541}.png}

\noindent\sphinxincludegraphics{{plantuml-7ea42fece8337980b63749c5f4d36671c83b3ab9}.png}

\noindent\sphinxincludegraphics{{plantuml-98bdd99e187fe94c08e127cced9c6e40171b78ce}.png}


\begin{savenotes}\sphinxatlongtablestart\begin{longtable}[c]{\X{1}{2}\X{1}{2}}
\hline

\endfirsthead

\multicolumn{2}{c}%
{\makebox[0pt]{\sphinxtablecontinued{\tablename\ \thetable{} -- continued from previous page}}}\\
\hline

\endhead

\hline
\multicolumn{2}{r}{\makebox[0pt][r]{\sphinxtablecontinued{Continued on next page}}}\\
\endfoot

\endlastfoot

\sphinxcode{\sphinxupquote{JK\_py.newclass}}
&

\\
\hline
\end{longtable}\sphinxatlongtableend\end{savenotes}


\chapter{Indices and tables}
\label{\detokenize{index:indices-and-tables}}\begin{itemize}
\item {} 
\DUrole{xref,std,std-ref}{genindex}

\item {} 
\DUrole{xref,std,std-ref}{modindex}

\item {} 
\DUrole{xref,std,std-ref}{search}

\end{itemize}



\renewcommand{\indexname}{Index}
\printindex
\end{document}